% Options for packages loaded elsewhere
\PassOptionsToPackage{unicode}{hyperref}
\PassOptionsToPackage{hyphens}{url}
%
\documentclass[
]{article}
\usepackage{amsmath,amssymb}
\usepackage{lmodern}
\usepackage{ifxetex,ifluatex}
\ifnum 0\ifxetex 1\fi\ifluatex 1\fi=0 % if pdftex
  \usepackage[T1]{fontenc}
  \usepackage[utf8]{inputenc}
  \usepackage{textcomp} % provide euro and other symbols
\else % if luatex or xetex
  \usepackage{unicode-math}
  \defaultfontfeatures{Scale=MatchLowercase}
  \defaultfontfeatures[\rmfamily]{Ligatures=TeX,Scale=1}
\fi
% Use upquote if available, for straight quotes in verbatim environments
\IfFileExists{upquote.sty}{\usepackage{upquote}}{}
\IfFileExists{microtype.sty}{% use microtype if available
  \usepackage[]{microtype}
  \UseMicrotypeSet[protrusion]{basicmath} % disable protrusion for tt fonts
}{}
\makeatletter
\@ifundefined{KOMAClassName}{% if non-KOMA class
  \IfFileExists{parskip.sty}{%
    \usepackage{parskip}
  }{% else
    \setlength{\parindent}{0pt}
    \setlength{\parskip}{6pt plus 2pt minus 1pt}}
}{% if KOMA class
  \KOMAoptions{parskip=half}}
\makeatother
\usepackage{xcolor}
\IfFileExists{xurl.sty}{\usepackage{xurl}}{} % add URL line breaks if available
\IfFileExists{bookmark.sty}{\usepackage{bookmark}}{\usepackage{hyperref}}
\hypersetup{
  pdftitle={Trabajo 2. Decisión Multicriterio},
  pdfauthor={Marta Venegas Pardo},
  hidelinks,
  pdfcreator={LaTeX via pandoc}}
\urlstyle{same} % disable monospaced font for URLs
\usepackage[margin=1in]{geometry}
\usepackage{color}
\usepackage{fancyvrb}
\newcommand{\VerbBar}{|}
\newcommand{\VERB}{\Verb[commandchars=\\\{\}]}
\DefineVerbatimEnvironment{Highlighting}{Verbatim}{commandchars=\\\{\}}
% Add ',fontsize=\small' for more characters per line
\usepackage{framed}
\definecolor{shadecolor}{RGB}{248,248,248}
\newenvironment{Shaded}{\begin{snugshade}}{\end{snugshade}}
\newcommand{\AlertTok}[1]{\textcolor[rgb]{0.94,0.16,0.16}{#1}}
\newcommand{\AnnotationTok}[1]{\textcolor[rgb]{0.56,0.35,0.01}{\textbf{\textit{#1}}}}
\newcommand{\AttributeTok}[1]{\textcolor[rgb]{0.77,0.63,0.00}{#1}}
\newcommand{\BaseNTok}[1]{\textcolor[rgb]{0.00,0.00,0.81}{#1}}
\newcommand{\BuiltInTok}[1]{#1}
\newcommand{\CharTok}[1]{\textcolor[rgb]{0.31,0.60,0.02}{#1}}
\newcommand{\CommentTok}[1]{\textcolor[rgb]{0.56,0.35,0.01}{\textit{#1}}}
\newcommand{\CommentVarTok}[1]{\textcolor[rgb]{0.56,0.35,0.01}{\textbf{\textit{#1}}}}
\newcommand{\ConstantTok}[1]{\textcolor[rgb]{0.00,0.00,0.00}{#1}}
\newcommand{\ControlFlowTok}[1]{\textcolor[rgb]{0.13,0.29,0.53}{\textbf{#1}}}
\newcommand{\DataTypeTok}[1]{\textcolor[rgb]{0.13,0.29,0.53}{#1}}
\newcommand{\DecValTok}[1]{\textcolor[rgb]{0.00,0.00,0.81}{#1}}
\newcommand{\DocumentationTok}[1]{\textcolor[rgb]{0.56,0.35,0.01}{\textbf{\textit{#1}}}}
\newcommand{\ErrorTok}[1]{\textcolor[rgb]{0.64,0.00,0.00}{\textbf{#1}}}
\newcommand{\ExtensionTok}[1]{#1}
\newcommand{\FloatTok}[1]{\textcolor[rgb]{0.00,0.00,0.81}{#1}}
\newcommand{\FunctionTok}[1]{\textcolor[rgb]{0.00,0.00,0.00}{#1}}
\newcommand{\ImportTok}[1]{#1}
\newcommand{\InformationTok}[1]{\textcolor[rgb]{0.56,0.35,0.01}{\textbf{\textit{#1}}}}
\newcommand{\KeywordTok}[1]{\textcolor[rgb]{0.13,0.29,0.53}{\textbf{#1}}}
\newcommand{\NormalTok}[1]{#1}
\newcommand{\OperatorTok}[1]{\textcolor[rgb]{0.81,0.36,0.00}{\textbf{#1}}}
\newcommand{\OtherTok}[1]{\textcolor[rgb]{0.56,0.35,0.01}{#1}}
\newcommand{\PreprocessorTok}[1]{\textcolor[rgb]{0.56,0.35,0.01}{\textit{#1}}}
\newcommand{\RegionMarkerTok}[1]{#1}
\newcommand{\SpecialCharTok}[1]{\textcolor[rgb]{0.00,0.00,0.00}{#1}}
\newcommand{\SpecialStringTok}[1]{\textcolor[rgb]{0.31,0.60,0.02}{#1}}
\newcommand{\StringTok}[1]{\textcolor[rgb]{0.31,0.60,0.02}{#1}}
\newcommand{\VariableTok}[1]{\textcolor[rgb]{0.00,0.00,0.00}{#1}}
\newcommand{\VerbatimStringTok}[1]{\textcolor[rgb]{0.31,0.60,0.02}{#1}}
\newcommand{\WarningTok}[1]{\textcolor[rgb]{0.56,0.35,0.01}{\textbf{\textit{#1}}}}
\usepackage{longtable,booktabs,array}
\usepackage{calc} % for calculating minipage widths
% Correct order of tables after \paragraph or \subparagraph
\usepackage{etoolbox}
\makeatletter
\patchcmd\longtable{\par}{\if@noskipsec\mbox{}\fi\par}{}{}
\makeatother
% Allow footnotes in longtable head/foot
\IfFileExists{footnotehyper.sty}{\usepackage{footnotehyper}}{\usepackage{footnote}}
\makesavenoteenv{longtable}
\usepackage{graphicx}
\makeatletter
\def\maxwidth{\ifdim\Gin@nat@width>\linewidth\linewidth\else\Gin@nat@width\fi}
\def\maxheight{\ifdim\Gin@nat@height>\textheight\textheight\else\Gin@nat@height\fi}
\makeatother
% Scale images if necessary, so that they will not overflow the page
% margins by default, and it is still possible to overwrite the defaults
% using explicit options in \includegraphics[width, height, ...]{}
\setkeys{Gin}{width=\maxwidth,height=\maxheight,keepaspectratio}
% Set default figure placement to htbp
\makeatletter
\def\fps@figure{htbp}
\makeatother
\setlength{\emergencystretch}{3em} % prevent overfull lines
\providecommand{\tightlist}{%
  \setlength{\itemsep}{0pt}\setlength{\parskip}{0pt}}
\setcounter{secnumdepth}{-\maxdimen} % remove section numbering
\usepackage{booktabs}
\usepackage{longtable}
\usepackage{array}
\usepackage{multirow}
\usepackage{wrapfig}
\usepackage{float}
\usepackage{colortbl}
\usepackage{pdflscape}
\usepackage{tabu}
\usepackage{threeparttable}
\usepackage{threeparttablex}
\usepackage[normalem]{ulem}
\usepackage{makecell}
\usepackage{xcolor}
\ifluatex
  \usepackage{selnolig}  % disable illegal ligatures
\fi

\title{Trabajo 2. Decisión Multicriterio}
\usepackage{etoolbox}
\makeatletter
\providecommand{\subtitle}[1]{% add subtitle to \maketitle
  \apptocmd{\@title}{\par {\large #1 \par}}{}{}
}
\makeatother
\subtitle{Teoría de la decisión}
\author{Marta Venegas Pardo}
\date{11/9/2021}

\begin{document}
\maketitle

{
\setcounter{tocdepth}{4}
\tableofcontents
}
\begin{Shaded}
\begin{Highlighting}[]
\FunctionTok{source}\NormalTok{(}\StringTok{"ScriptsNecesarios/teoriadecision\_funciones\_multicriterio.R"}\NormalTok{)}
\CommentTok{\#source("ScriptsNecesarios/teoriadecision\_funciones\_multicriterio\_diagram.R")}
\FunctionTok{source}\NormalTok{(}\StringTok{"ScriptsNecesarios/teoriadecision\_funciones\_multicriterio\_utiles.R"}\NormalTok{)}
\end{Highlighting}
\end{Shaded}

\hypertarget{enunciado}{%
\subsection{Enunciado}\label{enunciado}}

Luis hace su último año de residencia de anestesia en el hospital
General Universitario de Ciudad Real, pero tiene que ir un mes a
trabajar en el hospital Asepeyo en un pueblo a las afueras de Madrid,
por lo que va a vivirá durante un mes en Madrid Capital, pero debe ir y
volver a Ciudad Real 3 veces para hacer guardias.

Debe elegir un medio de transporte entre tres posibles alternativas:

\begin{itemize}
\item
  Alquilar una moto en Madrid para ese mes, que sería un total de 220\$
  más gasolina.
\item
  Compar un patinete eléctrico (200 euros) y sacar un bono de metro que
  le permita ir y volver al hospital desde el centro por 28 euros.
\end{itemize}

Para estas dos opciones, necesitaría sacar un abono de 10 viajes de
Renfe para ir y volver a Ciudad Real, por un importe de 195 euros.

\begin{itemize}
\tightlist
\item
  Llevar su coche, con el que puede ir al hospital desde el centro y a
  Ciudad Real cuando necesite hacer las guardias. Si escoge llevar su
  coche, tendría que alquilar una plaza de garaje por 200 euros, además
  de pagar la gasolina.
\end{itemize}

Para tomar una decisión, Luis ha considerado tres criterios:

\begin{itemize}
\item
  Seguridad
\item
  Precio
\item
  Condiciones de viaje

  \begin{itemize}
  \tightlist
  \item
    Tiempo invertido en el/los trayecto
  \item
    Meteorología (frío/lluvia/nieve)
  \end{itemize}
\end{itemize}

Dentro de las condiciones de viaje, considera que es importante
considerar el tiempo que tardaría en llegar al hospital en función del
medio de transporte elegido y las condiciones meteorológicas, que le
harían tener que cambiar su itinerario.

\begin{itemize}
\tightlist
\item
  Matriz comparación entre Criterios
\end{itemize}

\begin{longtable}[]{@{}llll@{}}
\toprule
Criterios & Seguridad & Condiciones & Precio \\
\midrule
\endhead
Seguridad & 1 & 9 & 7 \\
Condiciones & 1/9 & 1 & 2 \\
Precio & 1/7 & 1/3 & 1 \\
\bottomrule
\end{longtable}

\begin{itemize}
\tightlist
\item
  Matriz de ponderación de importancias dentro del criterio CONDICIONES
  DEL VIAJE
\end{itemize}

\begin{longtable}[]{@{}lll@{}}
\toprule
Condiciones & Tiempo(tray) & Meteorología \\
\midrule
\endhead
Tiempo(tray) & 1 & 7 \\
Meteorología & 1/7 & 1 \\
\bottomrule
\end{longtable}

\begin{itemize}
\tightlist
\item
  Matriz comparación entre Alternativas según Criterios
\end{itemize}

\begin{longtable}[]{@{}llll@{}}
\toprule
Seguridad & Moto & Patín & Coche \\
\midrule
\endhead
Moto & 1 & 1/3 & 1/7 \\
Patín & 3 & 1 & 1/5 \\
Coche & 7 & 5 (4*) & 1 \\
\bottomrule
\end{longtable}

\begin{longtable}[]{@{}llll@{}}
\toprule
Precio & Moto & Patín & Coche \\
\midrule
\endhead
Moto & 1 & 1/3 & 7 \\
Patín & 3 & 1 & 5 \\
Coche & 1/7 & 1/5 & 1 \\
\bottomrule
\end{longtable}

\begin{itemize}
\tightlist
\item
  Matriz comparación entre Alternativas según Subcriterios
\end{itemize}

\begin{longtable}[]{@{}llll@{}}
\toprule
Meteorología & Moto & Patín & Coche \\
\midrule
\endhead
Moto & 1 & 1 & 1/7 \\
Patín & 1 & 1 & 1/7 \\
Coche & 7 & 7 & 1 \\
\bottomrule
\end{longtable}

\begin{longtable}[]{@{}llll@{}}
\toprule
Tiempo & Moto & Patín & Coche \\
\midrule
\endhead
Moto & 1 & 7 & 2 \\
Patín & 1/7 & 1 & 1/5 \\
Coche & 1/2 & 5 & 1 \\
\bottomrule
\end{longtable}

Para conseguir una ordenación de las alternativas y obtener la mejor,
tenemos la siguiente información

El decisor nos ha proporcionado los siguientes pesos preferenciales:
W=(0.5,0.25,0.25)

Como ya sabíamos, para tomar una decisión, Luis ha considerado tres
criterios relevantes para tomar la decisión de clasificar los tres
medios de transporte para viajar durante ese mes que está en Madrid.

\begin{itemize}
\tightlist
\item
  Seguridad (maximizar)
\item
  Precio (minimizar)
\item
  Condiciones de viaje (maximizar)
\end{itemize}

Ahora encontramos una table con información relevante proporcionada por
Luis para resolver el problema.

\begin{longtable}[]{@{}lllll@{}}
\toprule
Criterio & Min/Max & Tipo & Parámetros & Parámetros \\
\midrule
\endhead
Seguridad & Max & III & q=1 & \\
Precio & Min & I & q=10 & p=30 \\
Condiciones de viaje & Max & II & s=3 & p=5 \\
\bottomrule
\end{longtable}

La matriz de decisión viene recogida en la siguiente tabla:

\begin{longtable}[]{@{}llll@{}}
\toprule
A/C & C1: Seguridad & C2: Precio & C3: Condiciones \\
\midrule
\endhead
Moto & 80 & 415 & 90 \\
Patín & 250 & 423 & 100 \\
Coche & 500 & 300 & 80 \\
\bottomrule
\end{longtable}

Para aplicar el método electre, iniciar el proceso con \(\alpha=0.75\) y
d=(\(\infty\),55,\(\infty\)). Es decir, para que una alternativa supere
a otra en precio, debe haber como mínimo, una diferencia de 55 euros,
para cantidades inferiores, consideramos que ambas alternativas son
equivalentes.

Utilizar los métodos Electre y Promethee para ordenar las alternativas y
obtener la mejor decisión.

\hypertarget{soluciuxf3n-con-paquete-ahp}{%
\subsection{Solución con paquete
AHP}\label{soluciuxf3n-con-paquete-ahp}}

\begin{Shaded}
\begin{Highlighting}[]
\CommentTok{\#devtools::install\_github("gluc/ahp", build\_vignettes = TRUE)}
\CommentTok{\# ahp::RunGUI()}
\FunctionTok{library}\NormalTok{(ahp)}
\NormalTok{dtMio }\OtherTok{=} \FunctionTok{Load}\NormalTok{(}\StringTok{"trabajo2TD.ahp"}\NormalTok{)}
\end{Highlighting}
\end{Shaded}

\begin{Shaded}
\begin{Highlighting}[]
\FunctionTok{Visualize}\NormalTok{(dtMio)}
\end{Highlighting}
\end{Shaded}

\begin{verbatim}
## [1] "Node$fields will be deprecated in the next release. Please use Node$attributes instead."
## [1] "Node$fields will be deprecated in the next release. Please use Node$attributes instead."
## [1] "Node$fields will be deprecated in the next release. Please use Node$attributes instead."
## [1] "Node$fields will be deprecated in the next release. Please use Node$attributes instead."
## [1] "Node$fields will be deprecated in the next release. Please use Node$attributes instead."
## [1] "Node$fields will be deprecated in the next release. Please use Node$attributes instead."
## [1] "Node$fields will be deprecated in the next release. Please use Node$attributes instead."
## [1] "Node$fields will be deprecated in the next release. Please use Node$attributes instead."
## [1] "Node$fields will be deprecated in the next release. Please use Node$attributes instead."
## [1] "Node$fields will be deprecated in the next release. Please use Node$attributes instead."
## [1] "Node$fields will be deprecated in the next release. Please use Node$attributes instead."
## [1] "Node$fields will be deprecated in the next release. Please use Node$attributes instead."
## [1] "Node$fields will be deprecated in the next release. Please use Node$attributes instead."
## [1] "Node$fields will be deprecated in the next release. Please use Node$attributes instead."
## [1] "Node$fields will be deprecated in the next release. Please use Node$attributes instead."
## [1] "Node$fields will be deprecated in the next release. Please use Node$attributes instead."
## [1] "Node$fields will be deprecated in the next release. Please use Node$attributes instead."
## [1] "Node$fields will be deprecated in the next release. Please use Node$attributes instead."
\end{verbatim}

\hypertarget{pesos-globales.-criteriossubcriterios-y-alternativas}{%
\subsubsection{Pesos globales. Criterios/Subcriterios y
Alternativas}\label{pesos-globales.-criteriossubcriterios-y-alternativas}}

\begin{Shaded}
\begin{Highlighting}[]
\NormalTok{ahp}\SpecialCharTok{::}\FunctionTok{Calculate}\NormalTok{(dtMio)}
\NormalTok{ahp}\SpecialCharTok{::}\FunctionTok{AnalyzeTable}\NormalTok{(dtMio, }\AttributeTok{sort=}\StringTok{"orig"}\NormalTok{,}\AttributeTok{variable=}\StringTok{"priority"}\NormalTok{)}
\end{Highlighting}
\end{Shaded}

Priority

Moto

PatinElectrico

Coche

Inconsistency

{Decidir un medio de transporte en Madrid}

{100.0\%}

{}

{}

{}

{ 9.5\% }

{Seguridad }

{79.6\%}

{8.8\%}

{19.5\%}

{71.7\%}

{ 9.0\% }

{CondicionesViaje }

{12.1\%}

{}

{}

{}

{ 0.0\% }

{TiempoTrayecto }

{87.5\%}

{59.2\%}

{7.5\%}

{33.3\%}

{ 1.3\% }

{Meteorologia }

{12.5\%}

{11.1\%}

{11.1\%}

{77.8\%}

{ 0.0\% }

{Precio }

{8.3\%}

{8.1\%}

{18.8\%}

{73.1\%}

{ 6.2\% }

\hypertarget{tabla-para-la-interpretaciuxf3n}

{14.1\%}

{18.0\%}

{67.9\%}

{ 9.5\% }

{Seguridad }

{79.6\%}

{7.0\%}

{15.5\%}

{57.1\%}

{ 9.0\% }

{CondicionesViaje }

{12.1\%}

{6.4\%}

{1.0\%}

{4.7\%}

{ 0.0\% }

{TiempoTrayecto }

{10.6\%}

{6.3\%}

{0.8\%}

{3.5\%}

{ 1.3\% }

{Meteorologia }

{1.5\%}

{0.2\%}

{0.2\%}

{1.2\%}

{ 0.0\% }

{Precio }

{8.3\%}

{0.7\%}

{1.6\%}

{6.1\%}

{ 6.2\% }

\hypertarget{muxe9todo-promethee}{%
\section{Método Promethee}\label{muxe9todo-promethee}}

\hypertarget{promethee-i}{%
\subsection{Promethee I}\label{promethee-i}}

\begin{Shaded}
\begin{Highlighting}[]
\NormalTok{tabdec.X }\OtherTok{=} \FunctionTok{multicriterio.crea.matrizdecision}\NormalTok{(}
  \FunctionTok{c}\NormalTok{(}\DecValTok{80}\NormalTok{,}\SpecialCharTok{{-}}\NormalTok{(}\DecValTok{220}\SpecialCharTok{+}\DecValTok{195}\NormalTok{),}\DecValTok{90}\NormalTok{,}
   \DecValTok{350}\NormalTok{,}\SpecialCharTok{{-}}\NormalTok{(}\DecValTok{200}\SpecialCharTok{+}\DecValTok{28}\SpecialCharTok{+}\DecValTok{195}\NormalTok{),}\DecValTok{110}\NormalTok{,}
   \DecValTok{450}\NormalTok{,}\SpecialCharTok{{-}}\NormalTok{(}\DecValTok{200}\SpecialCharTok{+}\DecValTok{100}\NormalTok{),}\DecValTok{90}\NormalTok{),}
  \AttributeTok{numalternativas=}\DecValTok{3}\NormalTok{,}
  \AttributeTok{numcriterios=}\DecValTok{3}\NormalTok{,}
  \AttributeTok{v.nombresalt=}\FunctionTok{c}\NormalTok{(}\StringTok{"Moto"}\NormalTok{,}\StringTok{"Patinete E"}\NormalTok{,}\StringTok{"Coche"}\NormalTok{),}
  \AttributeTok{v.nombrescri =} \FunctionTok{c}\NormalTok{(}\StringTok{"Seguridad"}\NormalTok{,}\StringTok{"Precio"}\NormalTok{,}\StringTok{"Condiciones"}\NormalTok{)}
                                                    
\NormalTok{)}
\NormalTok{tabdec.X}
\end{Highlighting}
\end{Shaded}

\begin{verbatim}
##            Seguridad Precio Condiciones
## Moto              80   -415          90
## Patinete E       350   -423         110
## Coche            450   -300          90
\end{verbatim}

\begin{Shaded}
\begin{Highlighting}[]
\NormalTok{wi}\OtherTok{\textless{}{-}}\FunctionTok{c}\NormalTok{(}\DecValTok{2}\SpecialCharTok{/}\DecValTok{5}\NormalTok{,}\DecValTok{1}\SpecialCharTok{/}\DecValTok{5}\NormalTok{,}\DecValTok{2}\SpecialCharTok{/}\DecValTok{5}\NormalTok{)}
\CommentTok{\# Le da más importancia a la seguridad y a las condiciones del viaje que al precio}
\NormalTok{salM}\OtherTok{=} \FunctionTok{multicriterio.metodo.promethee\_i}\NormalTok{(}
\NormalTok{  tabdec.X,}
  \AttributeTok{pesos.criterios =}\NormalTok{ wi, }\CommentTok{\# porque son 6 criterios, LOS PESOS DEBEN SUMAR 1}
  \AttributeTok{tab.fpref=}\FunctionTok{matrix}\NormalTok{(}\FunctionTok{c}\NormalTok{( }\DecValTok{2}\NormalTok{,}\DecValTok{1}\NormalTok{, }\DecValTok{0}\NormalTok{, }\DecValTok{0}\NormalTok{, }\CommentTok{\# nro ,qi,pi,si}
                      \DecValTok{3}\NormalTok{,}\DecValTok{10}\NormalTok{,}\DecValTok{30}\NormalTok{,}\DecValTok{0}\NormalTok{,}
                      \DecValTok{1}\NormalTok{,}\DecValTok{3}\NormalTok{, }\DecValTok{5}\NormalTok{, }\DecValTok{0}\NormalTok{),}
                   \AttributeTok{ncol=}\DecValTok{4}\NormalTok{,}
                   \AttributeTok{byrow=}\ConstantTok{TRUE}\NormalTok{))}

\NormalTok{  tab.fpref}\OtherTok{=}\FunctionTok{matrix}\NormalTok{(}\FunctionTok{c}\NormalTok{( }\DecValTok{2}\NormalTok{,}\DecValTok{1}\NormalTok{, }\DecValTok{0}\NormalTok{, }\DecValTok{0}\NormalTok{, }\CommentTok{\# nro ,qi,pi,si}
                      \DecValTok{3}\NormalTok{,}\DecValTok{10}\NormalTok{,}\DecValTok{30}\NormalTok{,}\DecValTok{0}\NormalTok{,}
                      \DecValTok{1}\NormalTok{,}\DecValTok{3}\NormalTok{, }\DecValTok{5}\NormalTok{, }\DecValTok{0}\NormalTok{),}\AttributeTok{ncol=}\DecValTok{4}\NormalTok{,}
                   \AttributeTok{byrow=}\ConstantTok{TRUE}\NormalTok{)}
\NormalTok{salM}
\end{Highlighting}
\end{Shaded}

\begin{verbatim}
## $tabla.indices
##            Moto Patinete E Coche
## Moto        0.0 0.05333333   0.0
## Patinete E  0.8 0.00000000   0.4
## Coche       0.6 0.60000000   0.0
## 
## $vflujos.ent
##       Moto Patinete E      Coche 
## 0.05333333 1.20000000 1.20000000 
## 
## $vflujos.sal
##       Moto Patinete E      Coche 
##  1.4000000  0.6533333  0.4000000 
## 
## $tablarelacionsupera
##            Moto Patinete E Coche
## Moto        0.5        0.0   0.0
## Patinete E  1.0        0.5   0.0
## Coche       1.0        1.0   0.5
\end{verbatim}

Representar en un grafo:

\begin{Shaded}
\begin{Highlighting}[]
\NormalTok{qgraph}\SpecialCharTok{::}\FunctionTok{qgraph}\NormalTok{(salM}\SpecialCharTok{$}\NormalTok{tablarelacionsupera)}
\end{Highlighting}
\end{Shaded}

\includegraphics{Tarea2MartaVenegas_files/figure-latex/unnamed-chunk-8-1.pdf}

Ir en coche domina a todos y el patinete domina a la moto. La moto no
domina a ninguna alternativa.

\hypertarget{promethee-ii}{%
\subsection{Promethee II}\label{promethee-ii}}

\begin{Shaded}
\begin{Highlighting}[]
\NormalTok{salMPrometheeII}\OtherTok{=} \FunctionTok{multicriterio.metodo.promethee\_ii}\NormalTok{(}
\NormalTok{  tabdec.X,}
  \AttributeTok{pesos.criterios =}\NormalTok{ wi, }\CommentTok{\# porque son 6 criterios, LOS PESOS DEBEN SUMAR 1}
  \AttributeTok{tab.fpref=}\FunctionTok{matrix}\NormalTok{(}\FunctionTok{c}\NormalTok{( }\DecValTok{2}\NormalTok{,}\DecValTok{1}\NormalTok{, }\DecValTok{1}\NormalTok{, }\DecValTok{0}\NormalTok{, }\CommentTok{\# nro ,qi,pi,si}
                      \DecValTok{3}\NormalTok{,}\DecValTok{10}\NormalTok{,}\DecValTok{30}\NormalTok{,}\DecValTok{0}\NormalTok{,}
                      \DecValTok{1}\NormalTok{,}\DecValTok{3}\NormalTok{, }\DecValTok{5}\NormalTok{, }\DecValTok{0}\NormalTok{),}
                   \AttributeTok{ncol=}\DecValTok{4}\NormalTok{,}
                   \AttributeTok{byrow=}\ConstantTok{TRUE}\NormalTok{))}

\NormalTok{  tab.fpref}\OtherTok{=}\FunctionTok{matrix}\NormalTok{(}\FunctionTok{c}\NormalTok{( }\DecValTok{2}\NormalTok{,}\DecValTok{1}\NormalTok{, }\DecValTok{1}\NormalTok{, }\DecValTok{0}\NormalTok{, }\CommentTok{\# nro ,qi,pi,si}
                      \DecValTok{3}\NormalTok{,}\DecValTok{10}\NormalTok{,}\DecValTok{30}\NormalTok{,}\DecValTok{0}\NormalTok{,}
                      \DecValTok{1}\NormalTok{,}\DecValTok{3}\NormalTok{, }\DecValTok{5}\NormalTok{, }\DecValTok{0}\NormalTok{),}\AttributeTok{ncol=}\DecValTok{3}\NormalTok{,}
                   \AttributeTok{byrow=}\ConstantTok{TRUE}\NormalTok{)}
\NormalTok{salMPrometheeII}
\end{Highlighting}
\end{Shaded}

\begin{verbatim}
## $tabla.indices
##            Moto Patinete E Coche
## Moto        0.0 0.05333333   0.0
## Patinete E  0.8 0.00000000   0.4
## Coche       0.6 0.60000000   0.0
## 
## $vflujos.netos
##       Moto Patinete E      Coche 
## -1.3466667  0.5466667  0.8000000 
## 
## $tablarelacionsupera
##            Moto Patinete E Coche
## Moto        0.5        0.0   0.0
## Patinete E  1.0        0.5   0.0
## Coche       1.0        1.0   0.5
\end{verbatim}

Representar en un grafo:

\begin{Shaded}
\begin{Highlighting}[]
\NormalTok{qgraph}\SpecialCharTok{::}\FunctionTok{qgraph}\NormalTok{(salMPrometheeII}\SpecialCharTok{$}\NormalTok{tablarelacionsupera)}
\end{Highlighting}
\end{Shaded}

\includegraphics{Tarea2MartaVenegas_files/figure-latex/unnamed-chunk-10-1.pdf}

De nuevo, ir en coche domina a todos y el patinete domina a la moto. La
moto no domina a ninguna alternativa.

\begin{Shaded}
\begin{Highlighting}[]
\FunctionTok{order}\NormalTok{(salMPrometheeII}\SpecialCharTok{$}\NormalTok{vflujos.netos,}\AttributeTok{decreasing =}\NormalTok{ T)}
\end{Highlighting}
\end{Shaded}

\begin{verbatim}
## [1] 3 2 1
\end{verbatim}

Me dice que la alternativa 3 (ir en coche) es la mejor, seguido de ir en
patín y por último, la moto.

\hypertarget{representaciuxf3n-gruxe1fica-de-las-funciones-de-preferencia}{%
\subsection{Representación gráfica de las funciones de
preferencia}\label{representaciuxf3n-gruxe1fica-de-las-funciones-de-preferencia}}

\begin{Shaded}
\begin{Highlighting}[]
\NormalTok{fpref.criterio\_usual\_di }\OtherTok{=} \ControlFlowTok{function}\NormalTok{(di) \{}
              \CommentTok{\#di = vaj{-}vah;}
    \ControlFlowTok{if}\NormalTok{ (di }\SpecialCharTok{\textless{}=} \DecValTok{0}\NormalTok{) \{}
\NormalTok{      res}\OtherTok{=} \DecValTok{0}\NormalTok{;}
\NormalTok{      \} }\ControlFlowTok{else}\NormalTok{ \{}
\NormalTok{        res}\OtherTok{=} \DecValTok{1}\NormalTok{;}
\NormalTok{      \}}
       \FunctionTok{return}\NormalTok{(res) ;}
\NormalTok{  \}}
\end{Highlighting}
\end{Shaded}

Representación Gráfica de las funciones de preferencias

\begin{Shaded}
\begin{Highlighting}[]
\NormalTok{x}\OtherTok{=} \FunctionTok{seq}\NormalTok{(}\SpecialCharTok{{-}}\DecValTok{10}\NormalTok{,}\DecValTok{10}\NormalTok{ ,}\AttributeTok{length.out=}\DecValTok{100}\NormalTok{)}
\NormalTok{y}\OtherTok{=} \FunctionTok{sapply}\NormalTok{ (x,fpref.criterio\_usual\_di)}
\FunctionTok{plot}\NormalTok{ (x,y, }\AttributeTok{type=}\StringTok{"l"}\NormalTok{, }\AttributeTok{col=}\StringTok{"blue"}\NormalTok{ ,}\AttributeTok{main=} \StringTok{"Criterio Usua: F1 "}\NormalTok{)}
\end{Highlighting}
\end{Shaded}

\includegraphics{Tarea2MartaVenegas_files/figure-latex/unnamed-chunk-13-1.pdf}

\begin{Shaded}
\begin{Highlighting}[]
\NormalTok{fpref.cuasi\_criterio\_di }\OtherTok{=} \ControlFlowTok{function}\NormalTok{(di,qi) \{}
              \CommentTok{\#di = vaj{-}vah;}
    \ControlFlowTok{if}\NormalTok{ (di }\SpecialCharTok{\textless{}=}\NormalTok{ qi) \{}
\NormalTok{      res}\OtherTok{=} \DecValTok{0}\NormalTok{;}
\NormalTok{      \} }\ControlFlowTok{else}\NormalTok{ \{}
\NormalTok{        res}\OtherTok{=} \DecValTok{1}\NormalTok{;}
\NormalTok{      \}}
       \FunctionTok{return}\NormalTok{(res) ;}
\NormalTok{  \}}
\end{Highlighting}
\end{Shaded}

\begin{Shaded}
\begin{Highlighting}[]
\NormalTok{x}\OtherTok{=} \FunctionTok{seq}\NormalTok{(}\SpecialCharTok{{-}}\DecValTok{10}\NormalTok{,}\DecValTok{10}\NormalTok{ ,}\AttributeTok{length.out=}\DecValTok{100}\NormalTok{)}
\NormalTok{y}\OtherTok{=} \FunctionTok{sapply}\NormalTok{ (x,}\ControlFlowTok{function}\NormalTok{(xx) }\FunctionTok{fpref.cuasi\_criterio\_di}\NormalTok{(xx,}\AttributeTok{qi=}\DecValTok{2}\NormalTok{))}
\FunctionTok{plot}\NormalTok{ (x,y, }\AttributeTok{type=}\StringTok{"l"}\NormalTok{, }\AttributeTok{col=}\StringTok{"blue"}\NormalTok{ ,}\AttributeTok{main=} \StringTok{"Cuasi Criterio. Lineal: F2 (con qi=2)"}\NormalTok{)}
\end{Highlighting}
\end{Shaded}

\includegraphics{Tarea2MartaVenegas_files/figure-latex/unnamed-chunk-15-1.pdf}

\begin{Shaded}
\begin{Highlighting}[]
\NormalTok{fpref.criterio\_preflineal\_di }\OtherTok{=} \ControlFlowTok{function}\NormalTok{(di,pi) \{}
              \CommentTok{\#di = vaj{-}vah;}
    \ControlFlowTok{if}\NormalTok{ (di }\SpecialCharTok{\textless{}=} \DecValTok{0}\NormalTok{) \{}
\NormalTok{      res}\OtherTok{=} \DecValTok{0}\NormalTok{;}
\NormalTok{      \} }\ControlFlowTok{else} \ControlFlowTok{if}\NormalTok{ (di}\SpecialCharTok{\textgreater{}}\NormalTok{pi) \{}
\NormalTok{        res}\OtherTok{=} \DecValTok{1}\NormalTok{;}
\NormalTok{      \} }\ControlFlowTok{else}\NormalTok{ \{}
\NormalTok{        res}\OtherTok{=}\NormalTok{di}\SpecialCharTok{/}\NormalTok{pi;}
\NormalTok{      \}}
       \FunctionTok{return}\NormalTok{(res) ;}
\NormalTok{  \}}
\end{Highlighting}
\end{Shaded}

\begin{Shaded}
\begin{Highlighting}[]
\NormalTok{x}\OtherTok{=} \FunctionTok{seq}\NormalTok{(}\SpecialCharTok{{-}}\DecValTok{10}\NormalTok{,}\DecValTok{10}\NormalTok{ ,}\AttributeTok{length.out=}\DecValTok{100}\NormalTok{)}
\NormalTok{y}\OtherTok{=} \FunctionTok{sapply}\NormalTok{ (x,}\ControlFlowTok{function}\NormalTok{(xx) }\FunctionTok{fpref.criterio\_preflineal\_di}\NormalTok{(xx,}\AttributeTok{pi=}\DecValTok{2}\NormalTok{))}
\FunctionTok{plot}\NormalTok{ (x,y, }\AttributeTok{type=}\StringTok{"l"}\NormalTok{, }\AttributeTok{col=}\StringTok{"blue"}\NormalTok{ ,}\AttributeTok{main=} \StringTok{"Criterio}
\StringTok{      Pref. Lineal: F3 (con pi=2)"}\NormalTok{)}
\end{Highlighting}
\end{Shaded}

\includegraphics{Tarea2MartaVenegas_files/figure-latex/unnamed-chunk-17-1.pdf}

\begin{Shaded}
\begin{Highlighting}[]
\NormalTok{fpref.criterio\_nivel\_di }\OtherTok{=} \ControlFlowTok{function}\NormalTok{(di,qi,pi) \{}
              \CommentTok{\#di = vaj{-}vah;}
    \ControlFlowTok{if}\NormalTok{ (di }\SpecialCharTok{\textless{}=}\NormalTok{ qi) \{}
\NormalTok{      res}\OtherTok{=} \DecValTok{0}\NormalTok{;}
\NormalTok{      \} }\ControlFlowTok{else} \ControlFlowTok{if}\NormalTok{ (di}\SpecialCharTok{\textgreater{}}\NormalTok{pi) \{}
\NormalTok{        res}\OtherTok{=} \DecValTok{1}\NormalTok{;}
\NormalTok{      \} }\ControlFlowTok{else}\NormalTok{ \{}
\NormalTok{        res}\OtherTok{=}\FloatTok{0.5}\NormalTok{;}
\NormalTok{      \}}
       \FunctionTok{return}\NormalTok{(res) ;}
\NormalTok{  \}}
\end{Highlighting}
\end{Shaded}

\begin{Shaded}
\begin{Highlighting}[]
\NormalTok{x}\OtherTok{=} \FunctionTok{seq}\NormalTok{(}\SpecialCharTok{{-}}\DecValTok{10}\NormalTok{,}\DecValTok{10}\NormalTok{ ,}\AttributeTok{length.out=}\DecValTok{100}\NormalTok{)}
\NormalTok{y}\OtherTok{=} \FunctionTok{sapply}\NormalTok{ (x,}\ControlFlowTok{function}\NormalTok{(xx) }\FunctionTok{fpref.criterio\_nivel\_di}\NormalTok{(xx,}\AttributeTok{pi=}\DecValTok{4}\NormalTok{,}\AttributeTok{qi=}\DecValTok{2}\NormalTok{))}
\FunctionTok{plot}\NormalTok{ (x,y, }\AttributeTok{type=}\StringTok{"l"}\NormalTok{, }\AttributeTok{col=}\StringTok{"blue"}\NormalTok{ ,}\AttributeTok{main=} \StringTok{"Criterio}
\StringTok{      Nivel: F4 (con qi=2 y pi=4)"}\NormalTok{)}
\end{Highlighting}
\end{Shaded}

\includegraphics{Tarea2MartaVenegas_files/figure-latex/unnamed-chunk-19-1.pdf}

\begin{Shaded}
\begin{Highlighting}[]
\NormalTok{fpref.criterio\_preflineal\_indif\_di }\OtherTok{=} \ControlFlowTok{function}\NormalTok{(di,qi,pi) \{}
              \CommentTok{\#di = vaj{-}vah;}
    \ControlFlowTok{if}\NormalTok{ (di }\SpecialCharTok{\textless{}=}\NormalTok{ qi) \{}
\NormalTok{      res}\OtherTok{=} \DecValTok{0}\NormalTok{;}
\NormalTok{      \} }\ControlFlowTok{else} \ControlFlowTok{if}\NormalTok{ (di}\SpecialCharTok{\textgreater{}}\NormalTok{pi) \{}
\NormalTok{        res}\OtherTok{=} \DecValTok{1}\NormalTok{;}
\NormalTok{      \} }\ControlFlowTok{else}\NormalTok{ \{}
\NormalTok{        res}\OtherTok{=}\NormalTok{(di}\SpecialCharTok{{-}}\NormalTok{qi)}\SpecialCharTok{/}\NormalTok{(pi}\SpecialCharTok{{-}}\NormalTok{qi);}
\NormalTok{      \}}
       \FunctionTok{return}\NormalTok{(res) ;}
\NormalTok{  \}}
\end{Highlighting}
\end{Shaded}

\begin{Shaded}
\begin{Highlighting}[]
\NormalTok{x}\OtherTok{=} \FunctionTok{seq}\NormalTok{(}\SpecialCharTok{{-}}\DecValTok{10}\NormalTok{,}\DecValTok{10}\NormalTok{ ,}\AttributeTok{length.out=}\DecValTok{100}\NormalTok{)}
\NormalTok{y}\OtherTok{=} \FunctionTok{sapply}\NormalTok{ (x,}\ControlFlowTok{function}\NormalTok{(xx) }\FunctionTok{fpref.criterio\_preflineal\_indif\_di}\NormalTok{(xx,}\AttributeTok{pi=}\DecValTok{4}\NormalTok{,}\AttributeTok{qi=}\DecValTok{2}\NormalTok{))}
\FunctionTok{plot}\NormalTok{ (x,y, }\AttributeTok{type=}\StringTok{"l"}\NormalTok{, }\AttributeTok{col=}\StringTok{"blue"}\NormalTok{ ,}\AttributeTok{main=} \StringTok{"Criterio Pref. Lineal Indiferencia: F5 (con qi=2 y pi=4)"}\NormalTok{)}
\end{Highlighting}
\end{Shaded}

\includegraphics{Tarea2MartaVenegas_files/figure-latex/unnamed-chunk-21-1.pdf}

\hypertarget{muxe9todo-electre}{%
\section{Método Electre}\label{muxe9todo-electre}}

Método electre:

\begin{itemize}
\tightlist
\item
  D
\item
  Alpha
\item
  Pesos de comparación, podríamos ponerlos todos infinitos
\end{itemize}

\hypertarget{primera-iteraciuxf3n}{%
\subsection{Primera iteración}\label{primera-iteraciuxf3n}}

\begin{Shaded}
\begin{Highlighting}[]
\NormalTok{salElectre}\OtherTok{=}\FunctionTok{multicriterio.metodoELECTRE\_I}\NormalTok{(tabdec.X,}
                      \AttributeTok{pesos.criterios =} \FunctionTok{c}\NormalTok{(}\DecValTok{2}\SpecialCharTok{/}\DecValTok{5}\NormalTok{,}\DecValTok{1}\SpecialCharTok{/}\DecValTok{5}\NormalTok{,}\DecValTok{2}\SpecialCharTok{/}\DecValTok{5}\NormalTok{),}
                      \AttributeTok{nivel.concordancia.minimo.alpha =} \FloatTok{0.7}\NormalTok{,}
                      \AttributeTok{no.se.compensan =} \FunctionTok{c}\NormalTok{(}\ConstantTok{Inf}\NormalTok{,}\DecValTok{55}\NormalTok{,}\ConstantTok{Inf}\NormalTok{),}
                      \AttributeTok{que.alternativas =} \ConstantTok{TRUE}
\NormalTok{                                   )}
\FunctionTok{cat}\NormalTok{(}\StringTok{"El núcleo es: "}\NormalTok{,salElectre}\SpecialCharTok{$}\NormalTok{nucleo\_aprox)}
\end{Highlighting}
\end{Shaded}

\begin{verbatim}
## El núcleo es:  2 3
\end{verbatim}

Es decir, no conseguimos distinguir qué alternativa es mejor, si ir en
patinete o en coche.

Representación del grafo

\begin{Shaded}
\begin{Highlighting}[]
\NormalTok{qgraph}\SpecialCharTok{::}\FunctionTok{qgraph}\NormalTok{(salElectre}\SpecialCharTok{$}\NormalTok{relacion.dominante)}
\end{Highlighting}
\end{Shaded}

\includegraphics{Tarea2MartaVenegas_files/figure-latex/unnamed-chunk-23-1.pdf}

Vamos a realizar iteraciones de este método para ver si conseguimos
establecer una relación de preferencia entre un método y otro.

\hypertarget{segunda-iteraciuxf3n}{%
\subsection{Segunda iteración}\label{segunda-iteraciuxf3n}}

En la siguiente iteración:

\begin{Shaded}
\begin{Highlighting}[]
\NormalTok{salElectrei2}\OtherTok{=}\FunctionTok{multicriterio.metodoELECTRE\_I}\NormalTok{(tabdec.X,}
                      \AttributeTok{pesos.criterios =}\NormalTok{ wi,}
                      \AttributeTok{nivel.concordancia.minimo.alpha =} \FloatTok{0.7}\NormalTok{,}
                      \AttributeTok{no.se.compensan =} \FunctionTok{c}\NormalTok{(}\ConstantTok{Inf}\NormalTok{,}\DecValTok{55}\NormalTok{,}\ConstantTok{Inf}\NormalTok{),}
                      \AttributeTok{que.alternativas =} \FunctionTok{c}\NormalTok{(}\DecValTok{2}\NormalTok{,}\DecValTok{3}\NormalTok{))}
                                   
\NormalTok{salElectrei2}\SpecialCharTok{$}\NormalTok{nucleo\_aprox}
\end{Highlighting}
\end{Shaded}

\begin{verbatim}
## Patinete E      Coche 
##          1          2
\end{verbatim}

El núcleo es el mismo que antes. Tengo que bajar alpha

\begin{Shaded}
\begin{Highlighting}[]
\NormalTok{qgraph}\SpecialCharTok{::}\FunctionTok{qgraph}\NormalTok{(salElectrei2}\SpecialCharTok{$}\NormalTok{relacion.dominante)}
\end{Highlighting}
\end{Shaded}

\includegraphics{Tarea2MartaVenegas_files/figure-latex/unnamed-chunk-25-1.pdf}

Ambas alternativas siguen siendo equivalentes.

\hypertarget{tercera-iteracciuxf3n}{%
\subsection{Tercera iteracción}\label{tercera-iteracciuxf3n}}

En la siguiente iteración:

\begin{Shaded}
\begin{Highlighting}[]
\NormalTok{salElectrei3}\OtherTok{=}\FunctionTok{multicriterio.metodoELECTRE\_I}\NormalTok{(tabdec.X,}
                      \AttributeTok{pesos.criterios =}\NormalTok{ wi,}
                      \AttributeTok{nivel.concordancia.minimo.alpha =} \FloatTok{0.6}\NormalTok{, }\CommentTok{\# 0.68, 0.60}
                      \AttributeTok{no.se.compensan =} \FunctionTok{c}\NormalTok{(}\ConstantTok{Inf}\NormalTok{,}\DecValTok{55}\NormalTok{,}\ConstantTok{Inf}\NormalTok{),}
                      \AttributeTok{que.alternativas =} \FunctionTok{c}\NormalTok{(}\DecValTok{2}\NormalTok{,}\DecValTok{3}\NormalTok{))}
                                   
\NormalTok{salElectrei3}\SpecialCharTok{$}\NormalTok{nucleo\_aprox}
\end{Highlighting}
\end{Shaded}

\begin{verbatim}
## Coche 
##     2
\end{verbatim}

Ahora, el núcleo es la alternativa 2, es decir: ir en coche.

\begin{Shaded}
\begin{Highlighting}[]
\NormalTok{qgraph}\SpecialCharTok{::}\FunctionTok{qgraph}\NormalTok{(salElectrei3}\SpecialCharTok{$}\NormalTok{relacion.dominante)}
\end{Highlighting}
\end{Shaded}

\includegraphics{Tarea2MartaVenegas_files/figure-latex/unnamed-chunk-27-1.pdf}

Por tanto, según el método electre, la mejor alternativa es: IR EN
COCHE.

\hypertarget{cuxe1lculos-en-el-muxe9todo-electre-i-la-representaciuxf3n-de-las-tables-de-test-con-kableextra}{%
\subsection{Cálculos en el método electre I La representación de las
tables de test con
KableExtra}\label{cuxe1lculos-en-el-muxe9todo-electre-i-la-representaciuxf3n-de-las-tables-de-test-con-kableextra}}

\begin{Shaded}
\begin{Highlighting}[]
\FunctionTok{library}\NormalTok{(dplyr)}
\FunctionTok{library}\NormalTok{(kableExtra)}
\FunctionTok{library}\NormalTok{(stringr)}
\NormalTok{salke }\OtherTok{=} \FunctionTok{func\_ELECTRE\_Completo}\NormalTok{(salElectre)}
\end{Highlighting}
\end{Shaded}

\begin{Shaded}
\begin{Highlighting}[]
\NormalTok{salke}\SpecialCharTok{$}\NormalTok{MIndices}
\end{Highlighting}
\end{Shaded}

\begin{verbatim}
## $MIndices
##   Alts Inds    A1    A2    A3
## 1   A1   I+           2      
## 2   A1   I= 1,2,3           3
## 3   A1   I-         1,3   1,2
## 4   A2   I+   1,3           3
## 5   A2   I=       1,2,3      
## 6   A2   I-     2         1,2
## 7   A3   I+   1,2   1,2      
## 8   A3   I=     3       1,2,3
## 9   A3   I-           3      
## 
## $KE
## \begin{table}
## \centering
## \begin{tabular}{>{\raggedright\arraybackslash}p{4em}>{\raggedright\arraybackslash}p{4em}|>{\raggedright\arraybackslash}p{10em}>{\raggedright\arraybackslash}p{10em}>{\raggedright\arraybackslash}p{10em}}
## \toprule
## \multicolumn{1}{c}{\textcolor{blue}{Alts}} & \multicolumn{1}{c}{\textcolor{blue}{Inds}} & \multicolumn{1}{c}{\textcolor{blue}{A1}} & \multicolumn{1}{c}{\textcolor{blue}{A2}} & \multicolumn{1}{c}{\textcolor{blue}{A3}}\\
## \midrule
##  & \textcolor{red}{\textbf{\multicolumn{1}{c}{\cellcolor[HTML]{808080}{\textbf{I+}}}}} & \multicolumn{1}{c}{\cellcolor[HTML]{808080}{\textbf{}}} & \multicolumn{1}{c}{\cellcolor[HTML]{808080}{\textbf{2}}} & \multicolumn{1}{c}{\cellcolor[HTML]{808080}{\textbf{}}}\\
## \cmidrule{2-5}
##  & \textcolor{red}{\textbf{\multicolumn{1}{c}{\cellcolor[HTML]{B3B3B3}{\textbf{I=}}}}} & \multicolumn{1}{c}{\cellcolor[HTML]{B3B3B3}{\textbf{1,2,3}}} & \multicolumn{1}{c}{\cellcolor[HTML]{B3B3B3}{\textbf{}}} & \multicolumn{1}{c}{\cellcolor[HTML]{B3B3B3}{\textbf{3}}}\\
## \cmidrule{2-5}
## \multirow{-3}{4em}{\raggedright\arraybackslash \cellcolor{white}{\textcolor{blue}{\textbf{\multicolumn{1}{c}{\textbf{A1}}}}}} & \textcolor{red}{\textbf{\multicolumn{1}{c}{\cellcolor[HTML]{E6E6E6}{\textbf{I-}}}}} & \multicolumn{1}{c}{\cellcolor[HTML]{E6E6E6}{\textbf{}}} & \multicolumn{1}{c}{\cellcolor[HTML]{E6E6E6}{\textbf{1,3}}} & \multicolumn{1}{c}{\cellcolor[HTML]{E6E6E6}{\textbf{1,2}}}\\
## \cmidrule{1-5}
##  & \textcolor{red}{\textbf{\multicolumn{1}{c}{\cellcolor[HTML]{808080}{\textbf{I+}}}}} & \multicolumn{1}{c}{\cellcolor[HTML]{808080}{\textbf{1,3}}} & \multicolumn{1}{c}{\cellcolor[HTML]{808080}{\textbf{}}} & \multicolumn{1}{c}{\cellcolor[HTML]{808080}{\textbf{3}}}\\
## \cmidrule{2-5}
##  & \textcolor{red}{\textbf{\multicolumn{1}{c}{\cellcolor[HTML]{B3B3B3}{\textbf{I=}}}}} & \multicolumn{1}{c}{\cellcolor[HTML]{B3B3B3}{\textbf{}}} & \multicolumn{1}{c}{\cellcolor[HTML]{B3B3B3}{\textbf{1,2,3}}} & \multicolumn{1}{c}{\cellcolor[HTML]{B3B3B3}{\textbf{}}}\\
## \cmidrule{2-5}
## \multirow{-3}{4em}{\raggedright\arraybackslash \cellcolor{white}{\textcolor{blue}{\textbf{\multicolumn{1}{c}{\textbf{A2}}}}}} & \textcolor{red}{\textbf{\multicolumn{1}{c}{\cellcolor[HTML]{E6E6E6}{\textbf{I-}}}}} & \multicolumn{1}{c}{\cellcolor[HTML]{E6E6E6}{\textbf{2}}} & \multicolumn{1}{c}{\cellcolor[HTML]{E6E6E6}{\textbf{}}} & \multicolumn{1}{c}{\cellcolor[HTML]{E6E6E6}{\textbf{1,2}}}\\
## \cmidrule{1-5}
##  & \textcolor{red}{\textbf{\multicolumn{1}{c}{\cellcolor[HTML]{808080}{\textbf{I+}}}}} & \multicolumn{1}{c}{\cellcolor[HTML]{808080}{\textbf{1,2}}} & \multicolumn{1}{c}{\cellcolor[HTML]{808080}{\textbf{1,2}}} & \multicolumn{1}{c}{\cellcolor[HTML]{808080}{\textbf{}}}\\
## \cmidrule{2-5}
##  & \textcolor{red}{\textbf{\multicolumn{1}{c}{\cellcolor[HTML]{B3B3B3}{\textbf{I=}}}}} & \multicolumn{1}{c}{\cellcolor[HTML]{B3B3B3}{\textbf{3}}} & \multicolumn{1}{c}{\cellcolor[HTML]{B3B3B3}{\textbf{}}} & \multicolumn{1}{c}{\cellcolor[HTML]{B3B3B3}{\textbf{1,2,3}}}\\
## \cmidrule{2-5}
## \multirow{-3}{4em}{\raggedright\arraybackslash \cellcolor{white}{\textcolor{blue}{\textbf{\multicolumn{1}{c}{\textbf{A3}}}}}} & \textcolor{red}{\textbf{\multicolumn{1}{c}{\cellcolor[HTML]{E6E6E6}{\textbf{I-}}}}} & \multicolumn{1}{c}{\cellcolor[HTML]{E6E6E6}{\textbf{}}} & \multicolumn{1}{c}{\cellcolor[HTML]{E6E6E6}{\textbf{3}}} & \multicolumn{1}{c}{\cellcolor[HTML]{E6E6E6}{\textbf{}}}\\
## \bottomrule
## \multicolumn{5}{l}{\rule{0pt}{1em}ELECTRE: Conjunto de Índices}\\
## \end{tabular}
## \end{table}
\end{verbatim}

Comentarios respecto a la alternativa 1 A1: Ir en moto

\begin{itemize}
\tightlist
\item
  Gana a la la alternativa 2 (ir en patinete) en el criterio 2 (precio),
  por lo tanto, es peor que ir en patinete en los criterios 1 y 3
  (seguridad y condiciones de viaje)
\item
  Gana a la la alternativa 3 (ir en coche) en el criterio 3 (condiciones
  de viaje), y por tanto , es peor que ir en coche en los criteios 1 y 2
  (seguridad y precio)
\item
  No es igual a ninguna alternativa en ningún criterio
\end{itemize}

La alternativa 2, A2: Ir en patín

\begin{itemize}
\tightlist
\item
  A1 no supera a A2 en los criterios 1, 3 y 5
\item
  En los criterios 1,3,5 a1 peor a2
\end{itemize}

\begin{Shaded}
\begin{Highlighting}[]
\NormalTok{salke}\SpecialCharTok{$}\NormalTok{TConcordancia}
\end{Highlighting}
\end{Shaded}

\begin{verbatim}
## $MIndices
##   Alts Inds  A1   A2     A3
## 1   A1  Ijk 1.0 0.20 0.4000
## 2   A1 IGjk NaN 0.25 0.0000
## 3   A1   TC 0.0 0.00 0.0000
## 4   A2  Ijk 0.8 1.00 0.4000
## 5   A2 IGjk 4.0  NaN 0.6667
## 6   A2   TC 1.0 0.00 0.0000
## 7   A3  Ijk 1.0 0.60 1.0000
## 8   A3 IGjk Inf 1.50    NaN
## 9   A3   TC 1.0 0.00 0.0000
## 
## $KE
## <table class="table" style="margin-left: auto; margin-right: auto;border-bottom: 0;">
##  <thead>
##   <tr>
##    <th style="text-align:left;color: blue !important;text-align: center;"> Alts </th>
##    <th style="text-align:left;color: blue !important;text-align: center;"> Inds </th>
##    <th style="text-align:left;color: blue !important;text-align: center;"> A1 </th>
##    <th style="text-align:left;color: blue !important;text-align: center;"> A2 </th>
##    <th style="text-align:left;color: blue !important;text-align: center;"> A3 </th>
##   </tr>
##  </thead>
## <tbody>
##   <tr>
##    <td style="text-align:left;font-weight: bold;background-color: rgba(128, 128, 128, 0.2) !important;text-align: center;width: 4em; font-weight: bold;color: blue !important;background-color: white !important;"> A1 </td>
##    <td style="text-align:left;font-weight: bold;background-color: rgba(128, 128, 128, 0.2) !important;text-align: center;width: 4em; font-weight: bold;color: red !important;border-right:1px solid;"> Ijk </td>
##    <td style="text-align:left;font-weight: bold;background-color: rgba(128, 128, 128, 0.2) !important;text-align: center;width: 10em; "> 1 </td>
##    <td style="text-align:left;font-weight: bold;background-color: rgba(128, 128, 128, 0.2) !important;text-align: center;width: 10em; "> 0.2 </td>
##    <td style="text-align:left;font-weight: bold;background-color: rgba(128, 128, 128, 0.2) !important;text-align: center;width: 10em; "> 0.4 </td>
##   </tr>
##   <tr>
##    <td style="text-align:left;font-weight: bold;background-color: rgba(179, 179, 179, 0.2) !important;text-align: center;width: 4em; font-weight: bold;color: blue !important;background-color: white !important;"> A1 </td>
##    <td style="text-align:left;font-weight: bold;background-color: rgba(179, 179, 179, 0.2) !important;text-align: center;width: 4em; font-weight: bold;color: red !important;border-right:1px solid;"> IGjk </td>
##    <td style="text-align:left;font-weight: bold;background-color: rgba(179, 179, 179, 0.2) !important;text-align: center;width: 10em; "> NaN </td>
##    <td style="text-align:left;font-weight: bold;background-color: rgba(179, 179, 179, 0.2) !important;text-align: center;width: 10em; "> 0.25 </td>
##    <td style="text-align:left;font-weight: bold;background-color: rgba(179, 179, 179, 0.2) !important;text-align: center;width: 10em; "> 0 </td>
##   </tr>
##   <tr>
##    <td style="text-align:left;font-weight: bold;background-color: rgba(230, 230, 230, 0.2) !important;text-align: center;width: 4em; font-weight: bold;color: blue !important;background-color: white !important;"> A1 </td>
##    <td style="text-align:left;font-weight: bold;background-color: rgba(230, 230, 230, 0.2) !important;text-align: center;width: 4em; font-weight: bold;color: red !important;border-right:1px solid;"> TC </td>
##    <td style="text-align:left;font-weight: bold;background-color: rgba(230, 230, 230, 0.2) !important;text-align: center;width: 10em; "> \cellcolor{white}{\textcolor{black}{F}} </td>
##    <td style="text-align:left;font-weight: bold;background-color: rgba(230, 230, 230, 0.2) !important;text-align: center;width: 10em; "> \cellcolor{white}{\textcolor{black}{F}} </td>
##    <td style="text-align:left;font-weight: bold;background-color: rgba(230, 230, 230, 0.2) !important;text-align: center;width: 10em; "> \cellcolor{white}{\textcolor{black}{F}} </td>
##   </tr>
##   <tr>
##    <td style="text-align:left;font-weight: bold;background-color: rgba(128, 128, 128, 0.2) !important;text-align: center;width: 4em; font-weight: bold;color: blue !important;background-color: white !important;"> A2 </td>
##    <td style="text-align:left;font-weight: bold;background-color: rgba(128, 128, 128, 0.2) !important;text-align: center;width: 4em; font-weight: bold;color: red !important;border-right:1px solid;"> Ijk </td>
##    <td style="text-align:left;font-weight: bold;background-color: rgba(128, 128, 128, 0.2) !important;text-align: center;width: 10em; "> 0.8 </td>
##    <td style="text-align:left;font-weight: bold;background-color: rgba(128, 128, 128, 0.2) !important;text-align: center;width: 10em; "> 1 </td>
##    <td style="text-align:left;font-weight: bold;background-color: rgba(128, 128, 128, 0.2) !important;text-align: center;width: 10em; "> 0.4 </td>
##   </tr>
##   <tr>
##    <td style="text-align:left;font-weight: bold;background-color: rgba(179, 179, 179, 0.2) !important;text-align: center;width: 4em; font-weight: bold;color: blue !important;background-color: white !important;"> A2 </td>
##    <td style="text-align:left;font-weight: bold;background-color: rgba(179, 179, 179, 0.2) !important;text-align: center;width: 4em; font-weight: bold;color: red !important;border-right:1px solid;"> IGjk </td>
##    <td style="text-align:left;font-weight: bold;background-color: rgba(179, 179, 179, 0.2) !important;text-align: center;width: 10em; "> 4 </td>
##    <td style="text-align:left;font-weight: bold;background-color: rgba(179, 179, 179, 0.2) !important;text-align: center;width: 10em; "> NaN </td>
##    <td style="text-align:left;font-weight: bold;background-color: rgba(179, 179, 179, 0.2) !important;text-align: center;width: 10em; "> 0.6667 </td>
##   </tr>
##   <tr>
##    <td style="text-align:left;font-weight: bold;background-color: rgba(230, 230, 230, 0.2) !important;text-align: center;width: 4em; font-weight: bold;color: blue !important;background-color: white !important;"> A2 </td>
##    <td style="text-align:left;font-weight: bold;background-color: rgba(230, 230, 230, 0.2) !important;text-align: center;width: 4em; font-weight: bold;color: red !important;border-right:1px solid;"> TC </td>
##    <td style="text-align:left;font-weight: bold;background-color: rgba(230, 230, 230, 0.2) !important;text-align: center;width: 10em; "> \cellcolor{blue}{\textcolor{white}{T}} </td>
##    <td style="text-align:left;font-weight: bold;background-color: rgba(230, 230, 230, 0.2) !important;text-align: center;width: 10em; "> \cellcolor{white}{\textcolor{black}{F}} </td>
##    <td style="text-align:left;font-weight: bold;background-color: rgba(230, 230, 230, 0.2) !important;text-align: center;width: 10em; "> \cellcolor{white}{\textcolor{black}{F}} </td>
##   </tr>
##   <tr>
##    <td style="text-align:left;font-weight: bold;background-color: rgba(128, 128, 128, 0.2) !important;text-align: center;width: 4em; font-weight: bold;color: blue !important;background-color: white !important;"> A3 </td>
##    <td style="text-align:left;font-weight: bold;background-color: rgba(128, 128, 128, 0.2) !important;text-align: center;width: 4em; font-weight: bold;color: red !important;border-right:1px solid;"> Ijk </td>
##    <td style="text-align:left;font-weight: bold;background-color: rgba(128, 128, 128, 0.2) !important;text-align: center;width: 10em; "> 1 </td>
##    <td style="text-align:left;font-weight: bold;background-color: rgba(128, 128, 128, 0.2) !important;text-align: center;width: 10em; "> 0.6 </td>
##    <td style="text-align:left;font-weight: bold;background-color: rgba(128, 128, 128, 0.2) !important;text-align: center;width: 10em; "> 1 </td>
##   </tr>
##   <tr>
##    <td style="text-align:left;font-weight: bold;background-color: rgba(179, 179, 179, 0.2) !important;text-align: center;width: 4em; font-weight: bold;color: blue !important;background-color: white !important;"> A3 </td>
##    <td style="text-align:left;font-weight: bold;background-color: rgba(179, 179, 179, 0.2) !important;text-align: center;width: 4em; font-weight: bold;color: red !important;border-right:1px solid;"> IGjk </td>
##    <td style="text-align:left;font-weight: bold;background-color: rgba(179, 179, 179, 0.2) !important;text-align: center;width: 10em; "> Inf </td>
##    <td style="text-align:left;font-weight: bold;background-color: rgba(179, 179, 179, 0.2) !important;text-align: center;width: 10em; "> 1.5 </td>
##    <td style="text-align:left;font-weight: bold;background-color: rgba(179, 179, 179, 0.2) !important;text-align: center;width: 10em; "> NaN </td>
##   </tr>
##   <tr>
##    <td style="text-align:left;font-weight: bold;background-color: rgba(230, 230, 230, 0.2) !important;text-align: center;width: 4em; font-weight: bold;color: blue !important;background-color: white !important;"> A3 </td>
##    <td style="text-align:left;font-weight: bold;background-color: rgba(230, 230, 230, 0.2) !important;text-align: center;width: 4em; font-weight: bold;color: red !important;border-right:1px solid;"> TC </td>
##    <td style="text-align:left;font-weight: bold;background-color: rgba(230, 230, 230, 0.2) !important;text-align: center;width: 10em; "> \cellcolor{blue}{\textcolor{white}{T}} </td>
##    <td style="text-align:left;font-weight: bold;background-color: rgba(230, 230, 230, 0.2) !important;text-align: center;width: 10em; "> \cellcolor{white}{\textcolor{black}{F}} </td>
##    <td style="text-align:left;font-weight: bold;background-color: rgba(230, 230, 230, 0.2) !important;text-align: center;width: 10em; "> \cellcolor{white}{\textcolor{black}{F}} </td>
##   </tr>
## </tbody>
## <tfoot><tr><td style="padding: 0; " colspan="100%">
## <sup></sup> ELECTRE: Test Concordancia</td></tr></tfoot>
## </table>
\end{verbatim}

Notación:

\begin{itemize}
\tightlist
\item
  F= NO PASA EL TEST DE CONCORDANCIA
\item
  RESULTADO AZUL: PAREJAS QUE PASAN EL TEST DE CONCORDANCIA
\end{itemize}

Sólo pasan el test de corcondancia:

\begin{itemize}
\tightlist
\item
  A2SA1: Ir en patín supera a ir en moto
\item
  A3SA1: Ir en coche supera a ir en moto
\item
  A3SA2: Ir en coche supera a ir en patin
\end{itemize}

\begin{Shaded}
\begin{Highlighting}[]
\NormalTok{salke}\SpecialCharTok{$}\NormalTok{TDiscordancia}
\end{Highlighting}
\end{Shaded}

\begin{verbatim}
## $MIndices
##   Alts Inds    A1               A2               A3
## 1   A1   I-                    1,3              1,2
## 2   A1  Djk       270(Inf),20(Inf) 370(Inf),115(55)
## 3   A1   TD     1                1                0
## 4   A2   I-     2                               1,2
## 5   A2  Djk 8(55)                  100(Inf),123(55)
## 6   A2   TD     1                1                0
## 7   A3   I-                      3                 
## 8   A3  Djk  <NA>          20(Inf)                 
## 9   A3   TD     1                1                1
## 
## $KE
## \begin{table}
## \centering
## \begin{tabular}{>{\raggedright\arraybackslash}p{4em}>{\raggedright\arraybackslash}p{4em}|>{\raggedright\arraybackslash}p{10em}>{\raggedright\arraybackslash}p{10em}>{\raggedright\arraybackslash}p{10em}}
## \toprule
## \multicolumn{1}{c}{\textcolor{blue}{Alts}} & \multicolumn{1}{c}{\textcolor{blue}{Inds}} & \multicolumn{1}{c}{\textcolor{blue}{A1}} & \multicolumn{1}{c}{\textcolor{blue}{A2}} & \multicolumn{1}{c}{\textcolor{blue}{A3}}\\
## \midrule
##  & \textcolor{red}{\textbf{\multicolumn{1}{c}{\cellcolor[HTML]{808080}{\textbf{I-}}}}} & \multicolumn{1}{c}{\cellcolor[HTML]{808080}{\textbf{}}} & \multicolumn{1}{c}{\cellcolor[HTML]{808080}{\textbf{1,3}}} & \multicolumn{1}{c}{\cellcolor[HTML]{808080}{\textbf{1,2}}}\\
## \cmidrule{2-5}
##  & \textcolor{red}{\textbf{\multicolumn{1}{c}{\cellcolor[HTML]{B3B3B3}{\textbf{Djk}}}}} & \multicolumn{1}{c}{\cellcolor[HTML]{B3B3B3}{\textbf{}}} & \multicolumn{1}{c}{\cellcolor[HTML]{B3B3B3}{\textbf{270(Inf),20(Inf)}}} & \multicolumn{1}{c}{\cellcolor[HTML]{B3B3B3}{\textbf{370(Inf),115(55)}}}\\
## \cmidrule{2-5}
## \multirow{-3}{4em}{\raggedright\arraybackslash \cellcolor{white}{\textcolor{blue}{\textbf{\multicolumn{1}{c}{\textbf{A1}}}}}} & \textcolor{red}{\textbf{\multicolumn{1}{c}{\cellcolor[HTML]{E6E6E6}{\textbf{TD}}}}} & \multicolumn{1}{c}{\cellcolor[HTML]{E6E6E6}{\textbf{\textcolor{white}{T}}}} & \multicolumn{1}{c}{\cellcolor[HTML]{E6E6E6}{\textbf{\textcolor{white}{T}}}} & \multicolumn{1}{c}{\cellcolor[HTML]{E6E6E6}{\textbf{\textcolor{black}{F}}}}\\
## \cmidrule{1-5}
##  & \textcolor{red}{\textbf{\multicolumn{1}{c}{\cellcolor[HTML]{808080}{\textbf{I-}}}}} & \multicolumn{1}{c}{\cellcolor[HTML]{808080}{\textbf{2}}} & \multicolumn{1}{c}{\cellcolor[HTML]{808080}{\textbf{}}} & \multicolumn{1}{c}{\cellcolor[HTML]{808080}{\textbf{1,2}}}\\
## \cmidrule{2-5}
##  & \textcolor{red}{\textbf{\multicolumn{1}{c}{\cellcolor[HTML]{B3B3B3}{\textbf{Djk}}}}} & \multicolumn{1}{c}{\cellcolor[HTML]{B3B3B3}{\textbf{8(55)}}} & \multicolumn{1}{c}{\cellcolor[HTML]{B3B3B3}{\textbf{}}} & \multicolumn{1}{c}{\cellcolor[HTML]{B3B3B3}{\textbf{100(Inf),123(55)}}}\\
## \cmidrule{2-5}
## \multirow{-3}{4em}{\raggedright\arraybackslash \cellcolor{white}{\textcolor{blue}{\textbf{\multicolumn{1}{c}{\textbf{A2}}}}}} & \textcolor{red}{\textbf{\multicolumn{1}{c}{\cellcolor[HTML]{E6E6E6}{\textbf{TD}}}}} & \multicolumn{1}{c}{\cellcolor[HTML]{E6E6E6}{\textbf{\textcolor{white}{T}}}} & \multicolumn{1}{c}{\cellcolor[HTML]{E6E6E6}{\textbf{\textcolor{white}{T}}}} & \multicolumn{1}{c}{\cellcolor[HTML]{E6E6E6}{\textbf{\textcolor{black}{F}}}}\\
## \cmidrule{1-5}
##  & \textcolor{red}{\textbf{\multicolumn{1}{c}{\cellcolor[HTML]{808080}{\textbf{I-}}}}} & \multicolumn{1}{c}{\cellcolor[HTML]{808080}{\textbf{}}} & \multicolumn{1}{c}{\cellcolor[HTML]{808080}{\textbf{3}}} & \multicolumn{1}{c}{\cellcolor[HTML]{808080}{\textbf{}}}\\
## \cmidrule{2-5}
##  & \textcolor{red}{\textbf{\multicolumn{1}{c}{\cellcolor[HTML]{B3B3B3}{\textbf{Djk}}}}} & \multicolumn{1}{c}{\cellcolor[HTML]{B3B3B3}{\textbf{NA}}} & \multicolumn{1}{c}{\cellcolor[HTML]{B3B3B3}{\textbf{20(Inf)}}} & \multicolumn{1}{c}{\cellcolor[HTML]{B3B3B3}{\textbf{}}}\\
## \cmidrule{2-5}
## \multirow{-3}{4em}{\raggedright\arraybackslash \cellcolor{white}{\textcolor{blue}{\textbf{\multicolumn{1}{c}{\textbf{A3}}}}}} & \textcolor{red}{\textbf{\multicolumn{1}{c}{\cellcolor[HTML]{E6E6E6}{\textbf{TD}}}}} & \multicolumn{1}{c}{\cellcolor[HTML]{E6E6E6}{\textbf{\textcolor{white}{T}}}} & \multicolumn{1}{c}{\cellcolor[HTML]{E6E6E6}{\textbf{\textcolor{white}{T}}}} & \multicolumn{1}{c}{\cellcolor[HTML]{E6E6E6}{\textbf{\textcolor{white}{T}}}}\\
## \bottomrule
## \multicolumn{5}{l}{\rule{0pt}{1em}ELECTRE: Test Discordancia: vd=(Inf,55,Inf)}\\
## \end{tabular}
## \end{table}
\end{verbatim}

Notación:

\begin{itemize}
\tightlist
\item
  F= NO PASA EL TEST DE DISCORDANCIA
\item
  RESULTADO AZUL: PAREJAS QUE PASAN EL TEST DE DISCORDANCIA
\item
  LA DIAGONAL SE SUPERA ENTERA PERO NO VA A INFLUIR
\end{itemize}

Las alternativas que pasaban el test de concordancia también pasan el de
discordancia, son:

\begin{itemize}
\tightlist
\item
  A2SA1: Ir en patín supera a ir en moto
\item
  A3SA1: Ir en coche supera a ir en moto
\item
  A3SA2: Ir en coche supera a ir en patin
\end{itemize}

\begin{Shaded}
\begin{Highlighting}[]
\NormalTok{salke}\SpecialCharTok{$}\NormalTok{TSuperacion}\SpecialCharTok{$}\NormalTok{KE}
\end{Highlighting}
\end{Shaded}

\begin{table}
\centering
\begin{tabular}{>{\raggedright\arraybackslash}p{4em}>{\raggedright\arraybackslash}p{4em}|>{\raggedright\arraybackslash}p{10em}>{\raggedright\arraybackslash}p{10em}>{\raggedright\arraybackslash}p{10em}}
\toprule
\multicolumn{1}{c}{\textcolor{blue}{Alts}} & \multicolumn{1}{c}{\textcolor{blue}{Inds}} & \multicolumn{1}{c}{\textcolor{blue}{A1}} & \multicolumn{1}{c}{\textcolor{blue}{A2}} & \multicolumn{1}{c}{\textcolor{blue}{A3}}\\
\midrule
 & \textcolor{red}{\textbf{\multicolumn{1}{c}{\cellcolor[HTML]{808080}{\textbf{TC}}}}} & \multicolumn{1}{c}{\cellcolor[HTML]{808080}{\textbf{\textcolor{black}{F}}}} & \multicolumn{1}{c}{\cellcolor[HTML]{808080}{\textbf{\textcolor{black}{F}}}} & \multicolumn{1}{c}{\cellcolor[HTML]{808080}{\textbf{\textcolor{black}{F}}}}\\
\cmidrule{2-5}
 & \textcolor{red}{\textbf{\multicolumn{1}{c}{\cellcolor[HTML]{B3B3B3}{\textbf{TD}}}}} & \multicolumn{1}{c}{\cellcolor[HTML]{B3B3B3}{\textbf{\textcolor{white}{T}}}} & \multicolumn{1}{c}{\cellcolor[HTML]{B3B3B3}{\textbf{\textcolor{white}{T}}}} & \multicolumn{1}{c}{\cellcolor[HTML]{B3B3B3}{\textbf{\textcolor{black}{F}}}}\\
\cmidrule{2-5}
\multirow{-3}{4em}{\raggedright\arraybackslash \cellcolor{white}{\textcolor{blue}{\textbf{\multicolumn{1}{c}{\textbf{A1}}}}}} & \textcolor{red}{\textbf{\multicolumn{1}{c}{\cellcolor[HTML]{E6E6E6}{\textbf{RSup}}}}} & \multicolumn{1}{c}{\cellcolor[HTML]{E6E6E6}{\textbf{\textcolor{black}{F}}}} & \multicolumn{1}{c}{\cellcolor[HTML]{E6E6E6}{\textbf{\textcolor{black}{F}}}} & \multicolumn{1}{c}{\cellcolor[HTML]{E6E6E6}{\textbf{\textcolor{black}{F}}}}\\
\cmidrule{1-5}
 & \textcolor{red}{\textbf{\multicolumn{1}{c}{\cellcolor[HTML]{808080}{\textbf{TC}}}}} & \multicolumn{1}{c}{\cellcolor[HTML]{808080}{\textbf{\textcolor{white}{T}}}} & \multicolumn{1}{c}{\cellcolor[HTML]{808080}{\textbf{\textcolor{black}{F}}}} & \multicolumn{1}{c}{\cellcolor[HTML]{808080}{\textbf{\textcolor{black}{F}}}}\\
\cmidrule{2-5}
 & \textcolor{red}{\textbf{\multicolumn{1}{c}{\cellcolor[HTML]{B3B3B3}{\textbf{TD}}}}} & \multicolumn{1}{c}{\cellcolor[HTML]{B3B3B3}{\textbf{\textcolor{white}{T}}}} & \multicolumn{1}{c}{\cellcolor[HTML]{B3B3B3}{\textbf{\textcolor{white}{T}}}} & \multicolumn{1}{c}{\cellcolor[HTML]{B3B3B3}{\textbf{\textcolor{black}{F}}}}\\
\cmidrule{2-5}
\multirow{-3}{4em}{\raggedright\arraybackslash \cellcolor{white}{\textcolor{blue}{\textbf{\multicolumn{1}{c}{\textbf{A2}}}}}} & \textcolor{red}{\textbf{\multicolumn{1}{c}{\cellcolor[HTML]{E6E6E6}{\textbf{RSup}}}}} & \multicolumn{1}{c}{\cellcolor[HTML]{E6E6E6}{\textbf{\textcolor{white}{T}}}} & \multicolumn{1}{c}{\cellcolor[HTML]{E6E6E6}{\textbf{\textcolor{black}{F}}}} & \multicolumn{1}{c}{\cellcolor[HTML]{E6E6E6}{\textbf{\textcolor{black}{F}}}}\\
\cmidrule{1-5}
 & \textcolor{red}{\textbf{\multicolumn{1}{c}{\cellcolor[HTML]{808080}{\textbf{TC}}}}} & \multicolumn{1}{c}{\cellcolor[HTML]{808080}{\textbf{\textcolor{white}{T}}}} & \multicolumn{1}{c}{\cellcolor[HTML]{808080}{\textbf{\textcolor{black}{F}}}} & \multicolumn{1}{c}{\cellcolor[HTML]{808080}{\textbf{\textcolor{black}{F}}}}\\
\cmidrule{2-5}
 & \textcolor{red}{\textbf{\multicolumn{1}{c}{\cellcolor[HTML]{B3B3B3}{\textbf{TD}}}}} & \multicolumn{1}{c}{\cellcolor[HTML]{B3B3B3}{\textbf{\textcolor{white}{T}}}} & \multicolumn{1}{c}{\cellcolor[HTML]{B3B3B3}{\textbf{\textcolor{white}{T}}}} & \multicolumn{1}{c}{\cellcolor[HTML]{B3B3B3}{\textbf{\textcolor{white}{T}}}}\\
\cmidrule{2-5}
\multirow{-3}{4em}{\raggedright\arraybackslash \cellcolor{white}{\textcolor{blue}{\textbf{\multicolumn{1}{c}{\textbf{A3}}}}}} & \textcolor{red}{\textbf{\multicolumn{1}{c}{\cellcolor[HTML]{E6E6E6}{\textbf{RSup}}}}} & \multicolumn{1}{c}{\cellcolor[HTML]{E6E6E6}{\textbf{\textcolor{white}{T}}}} & \multicolumn{1}{c}{\cellcolor[HTML]{E6E6E6}{\textbf{\textcolor{black}{F}}}} & \multicolumn{1}{c}{\cellcolor[HTML]{E6E6E6}{\textbf{\textcolor{black}{F}}}}\\
\bottomrule
\multicolumn{5}{l}{\rule{0pt}{1em}ELECTRE: Relación Superación: alpha = 0.7, vd=(Inf,55,Inf)}\\
\end{tabular}
\end{table}

CONCLUSIÓN:

\begin{itemize}
\tightlist
\item
  A2SA1: Ir en patín supera a ir en moto
\item
  A3SA1: Ir en coche supera a ir en moto
\item
  A3SA2: Ir en coche supera a ir en patin
\end{itemize}

\begin{Shaded}
\begin{Highlighting}[]
\NormalTok{salke}\SpecialCharTok{$}\NormalTok{Grafo}
\end{Highlighting}
\end{Shaded}

\begin{verbatim}
##   De A
## 1  2 1
## 2  3 1
\end{verbatim}

\begin{Shaded}
\begin{Highlighting}[]
\NormalTok{qgraph}\SpecialCharTok{::}\FunctionTok{qgraph}\NormalTok{(salke}\SpecialCharTok{$}\NormalTok{Grafo)}
\end{Highlighting}
\end{Shaded}

\includegraphics{Tarea2MartaVenegas_files/figure-latex/unnamed-chunk-34-1.pdf}

El núcleo

\begin{Shaded}
\begin{Highlighting}[]
\NormalTok{salke}\SpecialCharTok{$}\NormalTok{Nucleo}
\end{Highlighting}
\end{Shaded}

\begin{verbatim}
## Patinete E      Coche 
##          2          3
\end{verbatim}

Por tanto, la mejor alternativa es la 3, ir en coche. Segundo, ir en
patinete y por último, ir en moto sería la última alternativa a elegir.

\hypertarget{versiones}{%
\section{VERSIONES}\label{versiones}}

\hypertarget{matriz-de-comparaciuxf3n-entre-criterios}{%
\subsection{Matriz de comparación entre
criterios}\label{matriz-de-comparaciuxf3n-entre-criterios}}

En la versión 1, teníamos la siguiente matriz de comparación entre
criterios:

\begin{longtable}[]{@{}llll@{}}
\toprule
Criterios & Seguridad & Condiciones & Precio \\
\midrule
\endhead
Seguridad & 1 & 9 & 7 \\
Condiciones & 1/9 & 1 & 3 \\
Precio & 1/7 & 1/3 & 1 \\
\bottomrule
\end{longtable}

Con estos datos, obtenemos una inconsistencia del 19.6\%. Por ello, y
trás haber revisado los pesos, hemos obtenido la siguiente matriz de
comparación:

\begin{longtable}[]{@{}llll@{}}
\toprule
Criterios & Seguridad & Condiciones & Precio \\
\midrule
\endhead
Seguridad & 1 & 9 & 7 \\
Condiciones & 1/9 & 1 & 2 \\
Precio & 1/7 & 1/2 & 1 \\
\bottomrule
\end{longtable}

Ahora la inconsistencia es menor del 10\%

\hypertarget{matriz-de-comparaciuxf3n-entre-alternativas-seguxfan-criterios}{%
\subsection{Matriz de comparación entre alternativas según
criterios}\label{matriz-de-comparaciuxf3n-entre-alternativas-seguxfan-criterios}}

En el criterio \emph{seguridad}, teníamos una inconsistencia del 9\%
inicialmente, con la siguiente matríz de comparaciones:

\begin{longtable}[]{@{}llll@{}}
\toprule
Seguridad & Moto & Patín & Coche \\
\midrule
\endhead
Moto & 1 & 1/3 & 1/7 \\
Patín & 3 & 1 & 1/5 \\
Coche & 7 & 5 (4*) & 1 \\
\bottomrule
\end{longtable}

Por tanto, cambiando la matriz obtenemos una inconsistencia del 5.1\%:

\begin{longtable}[]{@{}llll@{}}
\toprule
Seguridad & Moto & Patín & Coche \\
\midrule
\endhead
Moto & 1 & 1/3 & 1/7 \\
Patín & 3 & 1 & 1/4 \\
Coche & 7 & 4 & 1 \\
\bottomrule
\end{longtable}

\hypertarget{muxe9todo-electre-1}{%
\subsection{Método electre}\label{muxe9todo-electre-1}}

Inicialmente, con unos pesos \(w_i=(0.5,0.25,0.25\) y aplicando el
método Electre, directamente en la primera iteracción obteníamos que el
núcleo era la altertativa: \emph{ir en coche}. Esto ocurría a pesar de
probar con distintos valores iniciales de \(\alpha\). Por ello, probamos
con los pesos \(w_i=(0.4,0.3,0.3\) y seguía ocurriendo lo mismo.
Finalmente, elegimos los pesos siguientes \(w_i=(0.4,0.2,0.4\), donde al
menos teníamos que realizar dos iteraciones para poder elegir una
alternativa.

\end{document}
